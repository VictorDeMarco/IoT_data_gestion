\capitulo{5}{Aspectos relevantes del desarrollo del proyecto}

Este apartado recoge los aspectos más interesantes del desarrollo del proyecto y explica los detalles de mayor relevancia de las fases de análisis, diseño e implementación.

\section{Despliegue de una red LoRaWAN}
El primer paso en el desarrollo del proyecto consistió en la creación y configuración de una red LoRaWAN, con el objetivo principal de familiarizarme con el ecosistema de los sistemas IoT y, al mismo tiempo, establecer un flujo constante y fiable de datos enviados por un dispositivo IoT.

Estos datos serían posteriormente almacenados y estructurados en un dataset, que actuaría como pilar fundamental para el desarrollo y validación de las siguientes partes del proyecto. Por este motivo, era especialmente importante asegurar que esta primera parte se realizase de forma correcta.

A continuación voy a describir los diferentes pasos necesarios para el despliegue de una red LoRaWAN:

\subsection{Configuración del gateway Dragino}
La recepción y retransmisión de paquetes LoRaWAN en este proyecto se realizó a través del Gateway Dragino LPS8. Este gateway tiene como principal función recibir datos de dispositivos IoT LoRa--en el contexto de este proyecto, el dispositivo MerryIoT MS10-- y después transmitirlos a través de Internet a un Network Server, en este caso, The Things Network.

Para configurar dicho dispositivo, lo primero sería asegurarse de que esté conectado a una fuente de alimentación y que tenga acceso a internet. En este caso, se le ha conectado vía Ethernet al router de mi entorno de trabajo.

Lo siguiente sería acceder a la interfaz web de configuración del dispositivo, dicha interfaz web puede encontrarse en la dirección http://IPADDRESS:8000 donde IPADDRESS es la dirección IP asignada al gateway por el router al que esté conectado. Antes de poder configurar la interfaz web te solicitará ingresar usuario y contraseña.

User Name: root

Password: dragino

Una vez dentro, será necesario configurar dos apartados para asegurar el correcto funcionamiento de la futura conexión con TTN.

El primero referente a permitir la conexión del gateway con el Network Server.

\imagen{Gateway_config}{Configuración del gateway}{0.8}

El segundo referente a la configurar el correcto plan de frecuencia entre el sensor y el gateway.

\imagen{Gateway_frec}{Frecuencia del gateway}{0.8}
  
Si has seguido correctamente estos pasos ya solo queda registrar ambos dispositivos IoT en TTN y observar el flujo de datos entre ellos.


\subsection{Configuración y registro en The Things Network (TTN) }
En este proyecto se ha optado por utilizar The Things Stack Sandbox(TTSS) para monitorizar los dispositivos IoT, ya que permite registrar dispositivos y observar el tráfico de red en tiempo real de forma gratuita.

A continuación se explicará el proceso seguido para lograr registrar correctamente tanto el gateway como el sensor en TTSS.

Una vez tienes creada tu cuenta en TTSS, primero debes elegir la opción register gateway. Esta opción, lo primero que te va a pedir es el Gateway EUI (se puede obtener en la interfaz web del gateway, en la pestaña referente a la configuración LoRaWAN), nombrar el gateway como el usuario desee y, por último, establecer un plan de frecuencia concorde con el configurado en la interfaz web. 

Si has seguido los pasos correctamente en TTSS debería verse algo similar a esto:

\imagen{Gateway_status}{Estado del gateway}{0.5}

En TTSS es necesario crear una aplicación para poder gestionar los sensores. Solo se necesita elegir la opción Add Application  y escoger un ID único y un nombre a elección del usuario. 

Con la aplicación creada se selecciona la opción Add end device y se elige el método de añadir las especificaciones de forma manual. Lo siguiente es introducir el mismo plan de frecuencia que en el gateway y las versiones correspondientes de LoRaWAN y parámetros al sensor a utilizar, en el caso del sensor del proyecto dichas versiones son 
LoRaWAN Specification 1.0.4 y 
RP002 Regional Parameters 1.0.3. Por último te pide introducir  DevEUI, AppEUI, y AppKey, parámetros que vienen incluidos en la etiqueta del sensor o en la documentación del fabricante.

Si todo ha ido bien, ya podríamos empezar a observar el flujo de datos entre nuestros dispositivos IoT.

\subsubsection{Pruebas iniciales para observar el flujo de datos}
Gracias TTSS podemos ver como interactúan los dispositivos IoT en la consola Live data, aunque todavía no podemos sacar información relevante debido a que nos falta por implementar un payload formatter, un pequeño fragmento de código capaz de decodificar el payload que envía el sensor IoT al gateway.

El payload formatter es diferente para cada sensor y debe ser creado a partir del payload de ejemplo mostrado en el manual de usuario del sensor~\cite{MerryIoT:manual}.

\imagen{Formatter}{Codigo del payload formatter}{0.8}

Una vez aplicado el payload formatter la información útil que podemos sacar del mensaje enviado por el sensor se ve así:

\imagen{decoded_payload}{Payload decodificado}{0.8}


\section{Desarrollo del analizador y receptor de paquetes }
Una vez configurada la red LoRaWAN y comprobado el correcto flujo de datos entre los dispositivos, el siguiente paso era buscar cómo almacenar todos esos datos de forma segura para crear un dataset el cual analizar en el futuro.

Se estudiaron diversas formas de llevar a cabo este proceso, desde servicios web relacionados con bases de datos hasta almacenarlos de forma local, pero todas las opciones pasaban por configurar un webhook en TTSS.
\subsubsection{Webhook}

Primero voy a explicar qué es y comó se configura un webhook en TTSS:

Un webhook es un mecanismo de callback HTTP. Dicho mecanismo enviará automáticamente una petición HTTP POST a una URL específica cada vez que un dispositivo envíe datos (uplink).

En tu aplicación de TTSS, elege el apartado webhooks y selecciona Add webhook. Escribes un ID único y escoges JSON como formato. En la url base escribes la dirección que quieres que reciba los datos. En el caso del proyecto, como solo nos interesa recibir los datos (uplink), en el apartado Enabled event types marcamos Uplink message y añadimos /ttn. Así cada vez que el webhook quiera enviar un mensaje a la url base, lo hará al apartado /ttn. 

Si has seguido los pasos correctamente, el webhook debería verse así:

\imagen{webhook}{Imagen webhook TTSS}{1}


En este proyecto se optó por redirigir los mensajes a un servidor propio en vez de guardarlos en la nube, para facilitar el análisis que se llevaría a cabo posteriormente.

Para realizar este proceso se decidió usar un webhook de TTSS y Cloudflare Tunnel como método para exponer de forma segura el servidor local al exterior.

Gracias a realizarlo de esta forma, fue posible exponer el servidor Flask local sin necesidad de modificar la configuración del router, garantizar seguridad en la transmisión (ya que Cloudflare proporciona un túnel seguro y cifrado) y configurar de forma sencilla el flujo de datos. 

\subsubsection{Función de recibir y almacenar}

Una vez terminado todo este proceso, era necesario crear una aplicación web en Flask capaz de recibir y almacenar los datos. Se optó por que fueran almacenados en un archivo csv. A futuro, cuando ya se habían almacenado suficientes datos y se disponía de un dataset de tamaño considerable, se procedió a añadir la función de analizar los paquetes de datos recibidos, pero primero nos centraremos en las funciones de recibir y almacenar.

La aplicación Flask se diseñó para exponer un endpoint HTTP (/ttn) accesible públicamente gracias al túnel de Cloudflare. Este endpoint es el receptor directo de los paquetes LoRaWAN enviados desde TTSS.

Cada paquete recibido contiene una estructura JSON con múltiples parámetros. Los datos son extraídos del decoded payload proporcionado por TTN y estructurados en un diccionario interno para luego ser almacenados en un archivo CSV local como una nueva fila. Dicho archivo es el que será utilizado como dataset durante todo el proyecto.

\subsubsection{Elección de método para clasificar}

En dicho dataset los paquetes de datos tienen un atributo denominado estado el cual al comienzo del proyecto se asumía que su valor era "real". Cuando se disponían de suficientes paquetes como para tratar de automatizar un proceso capaz de analizar y determinar de forma correcta el valor del atributo estado de cada paquete, comencé a plantearme cuál sería el mejor método para realizar el análisis y la clasificación correspondiente.

Algunas de las opciones que fueron descartadas, pero que se plantearon son:

Modelos de aprendizaje automático supervisado: Se contempló el uso de algoritmos como árboles de decisión. Sin embargo, esta opción fue descartada ya que el dataset del que se disponía para entrenar el modelo no era el adecuado. Este dataset, creado a partir de los paquetes recibidos de un dispositivo IoT, tiene ciertas peculiaridades que imposibilitan el uso de esta técnica. Por poner un ejemplo, el modelo daba mucho peso al voltaje recibido a la hora de determinar si un paquete era real o falso. Como todos los paquetes del dataset habían sido enviados desde el mismo dispositivo, todos compartían el mismo voltaje. Por lo tanto, el modelo asumía incorrectamente que los paquetes reales debían tener un valor específico de voltaje, ignorando el resto de atributos del paquete, igual o más importantes. Dicho modelo podía llegar a clasificar como real un paquete que marcara 100 grados, siempre que su voltaje fuera 2.4, debido a las asociaciones incorrectas que estableció por contar con un dataset no lo suficientemente amplio ni variado, como requieren este tipo de modelos.

Detección de anomalías sin supervisión (clustering): Técnicas como k-means o algoritmos de aislamiento fueron consideradas como posibles herramientas para detectar desviaciones. No obstante, requerían una preconfiguración compleja y presentaban un alto riesgo de falsos positivos ante pequeñas variaciones legítimas.

Finalmente, se optó por un enfoque más sencillo, transparente y controlado: una combinación de reglas heurísticas definidas manualmente junto a un sistema complementario de reglas de asociación extraídas con ayuda de la librería mlxtend.

\subsubsection{Función de clasificar/analizar}

Las reglas heurísticas permitieron codificar conocimientos específicos sobre el comportamiento esperado del dispositivo, como los rangos normales de temperatura y humedad. En cambio, las reglas de asociación ofrecieron una forma de complementar este sistema con correlaciones frecuentemente observadas entre atributos en paquetes etiquetados previamente, como las condiciones de detección de presencia.

Este método ofrece ciertas ventajas tales como:
\begin{itemize}
    \item Fácil ajuste y mantenimiento: permite incorporar nuevas condiciones o relajar otras en función de nuevas observaciones.
    \item No requiere entrenamiento previo ni datasets amplios y variados.
    \item Robustez y eficacia suficiente para el entorno planteado por el proyecto.
\end{itemize}

Para obtener las reglas de asociación, se decidió tener en cuenta únicamente aquellas que superaran un 15\% de soporte, es decir, que la regla predijera correctamente al menos un 15\% de las instancias del dataset, y un 90\% de confianza, es decir, que de todas las instancias en las que la regla es aplicable, al menos sea capaz de predecir correctamente el 90\% de los casos.

\imagen{sup_conf}{Fragmento de código encargado de determinar el soporte y confianza}{0.8}

Tras ejecutar ese código, las reglas obtenidas son:

\imagen{reglas_obtenidas}{Reglas de asociación obtenidas}{0.8}

Si el tiempo de espera es largo, entonces No hay presencia.
Traducido a nuestro dataset, es si time\_since\_last\_event\_min es mayor que 1, entonces tamper\_detected es false. Esto se cumple en los paquetes reales.

Si el tiempo de espera es corto, entonces hay presencia.
Traducido a nuestro dataset es si time\_since\_last\_event\_min es menor o igual que 1 y tamper\_detected es true, el paquete es real.

Gracias a esas reglas podemos determinar que, si Hay presencia y tiempo de espera es largo, el paquete no va ser real.

Al principio, se podría interpretar que otra posible regla sería: si no hay presencia y el tiempo de espera es corto, entonces el paquete es falso. Sin embargo, esto no es aplicable, ya que si se observan las reglas de asociación obtenidas, la regla si no hay presencia, entonces el tiempo de espera es largo no aparece.
Esto se debe a que no cumple con el soporte o la confianza mínimos, lo cual indica que existen más casos de los permitidos en los que no hay presencia y, aun así, el tiempo de espera es corto.
Por lo tanto, no podemos asegurar que un paquete sin presencia vaya necesariamente a tener un tiempo de espera largo. Esto hace posible la existencia de paquetes reales con tiempo de espera corto y sin presencia, lo que desmiente la regla asumida inicialmente.


Las reglas heurísticas elegidas para acompañar a las reglas de asociación antes mencionadas son: 

\begin{itemize}
    \item El voltaje de la batería no puede superar 3V ni puede ser menor 2.4V como determina el manual de usuario del sensor.
    \item La temperatura medida no puede ser superior a 50 grados ni inferior a 0 grados.
    \item La humedad no puede ser superior a 100\% ni inferior a 0\%
    \item El atributo ocupado no puede ser True
\end{itemize}

Por último se añadió una última regla algo más compleja que las demás, ya que no depende solo del paquete recibido sino que depende también del último paquete real almacenado. Si ambos paquetes no detectan presencia, se comprueba que el tiempo que haya pasado entre ellos sea de aproximadamente una hora, que es el periodo de tiempo que tarda en enviar un nuevo paquete el sensor si no detecta ninguna presencia.

\imagen{Regla_f}{Fragmento de código referente a la ultima regla añadida}{0.8}

Por ultimo, una vez que el código evalúa el nuevo paquete con todas las reglas mencionadas, determina si es un paquete infectado o si es un paquete real y procede a almacenar en el archivo del dataset con el atributo estado determinado tras el análisis.


\section{Desarrollo de la aplicación web enfocada al usuario }

Una vez terminada la infraestructura necesaria para la recepción y análisis automático de los datos recibidos, el siguiente paso del proyecto consistió en diseñar y desarrollar una segunda aplicación web, esta vez centrada en la interacción directa con el usuario. Esta nueva interfaz tiene como objetivo proporcionar herramientas visuales e intuitivas que permitan gestionar, visualizar, analizar y simular datos provenientes del dispositivo IoT.

Esta aplicación web se desarrolló también con Flask y se desarrolló en torno a cumplir una serie de funcionalidades que permiten al usuario:

\begin{itemize}
    \item Gestionar distintas cuentas diferenciadas entre si.
    \item Visualizar las gráficas de temperatura y humedad basadas en los datos almacenados.
    \item Gestionar distintos ficheros CSV (subir, eliminar, ver, descargar o analizar).
    \item Enviar paquetes personalizados o aleatorios al receptor para simular tráfico IoT.
\end{itemize}


\subsubsection{Creación de usuarios}
Lo primero que te encuentras al intentar acceder a esta aplicación web es una pestaña de login, esto se debe a que la aplicación cuenta con un login segurizado, que se caracteriza por no permitir al usuario acceder a ninguna funcionalidad de la aplicación, hasta que no haya iniciado sesión, permitiéndole solo iniciar sesión, registrarse o recuperar su contraseña. Evitando así posibles problemas generados al intentar usar las funcionalidades de la aplicación sin tener una cuenta asociada.  

Al principio del proyecto se planteó la opción de desarrollar la aplicación sin la posibilidad de que hubiera diferenciación entre usuarios, pero esto provocaba más problemas que soluciones al complicar mucho más las funcionalidades referentes a la gestión de archivos.

Entonces se decidió implementar una pequeña base de datos encargada de almacenar a los usuarios con sus contraseñas y así separar las acciones de un usuario de las de otro.


\textbf{Inicio de Sesión}

Referente al inicio de sesión , la aplicación obliga al usuario a rellenar ambos campos (Usuario, Contraseña), para evitar fallas de seguridad y notifica al usuario si, al intentar iniciar sesión, el usuario introducido no existe o no concuerda con la contraseña introducida.

Si no se desea crear una cuenta, se puede iniciar sesión con Usuario: admin, contraseña:admin.

\textbf{Registro}

Referente al registro, aclarar que no se permite que existan dos usuarios con el mismo nombre. Se notifica al usuario si intenta registrarse con un usuario ya existente. Una vez creado el nuevo usuario, se le asigna una carpeta dentro del directorio csv del proyecto, donde se guardarán sus archivos futuros.

\textbf{Recuperar Contraseña}

Se planteó la posibilidad de que el usuario se olvidara de su contraseña. Para solucionar este caso, se optó por pedir al usuario que aportara el nombre de uno de los ficheros csv que se encuentre en su carpeta personal. Si lo aporta, se le permite restablecer la contraseña.


\subsubsection{Visualizar gráficas}

Esta fue la primera funcionalidad que se planteó para la aplicación web ya que mi objetivo inicial era simplemente mostrar al usuario las gráficas referentes a los datos recopilados (Temperatura y Humedad) por el sensor IoT.

Con el transcurso del desarrollo del proyecto se optó por permitir al usuario interactuar mucho más con esta funcionalidad, desde que pueda elegir si solo mostrar en la gráfica los paquetes con el atributo estado igual a real, o permitir que visualice en la gráfica cualquier archivo csv que haya subido a la aplicación.

Todo este apartado se desarrolló gracias Chart.js que permite generar gráficos interactivos en HTML.


\imagen{Visualizar_graficas}{Interfaz gráfica referente a visualizar gráficas}{0.8}

En la siguiente imagen se observa la posibilidad de que el usuario visualice las gráficas correspondientes a un archivo csv propio. 

\imagen{Visualizar_diff}{Visualizar gráficas de un archivo diferente al dataset}{0.8}


\subsubsection{Gestionar ficheros CSV}

Esta funcionalidad es la más extensa del proyecto debido a la variedad de opciones que permite al usuario realizar.

Explicaré brevemente las funcionalidades más simples:

\begin{itemize}
    \item Visualizar contenido del archivo.
    \item Aplicar el archivo para que sea visible en el apartado de las gráficas.
    \item Eliminar el archivo.
    \item Descargar el archivo.
\end{itemize}

Las siguientes dos funcionalidades son analizar fichero y añadir fichero. Aunque puedan sonar similares, son funcionalidades completamente distintas.

\textbf{Añadir fichero}

Una vez implementada la posibilidad de diferenciar entre usuarios, era necesario implementar la posibilidad de que cada usuario pudiera interactuar con los archivos csv que deseara. Por ello, se desarrolló esta funcionalidad, la cual permite al usuario añadir un fichero que soporte la aplicación (Misma estructura que el fichero csv webhook\_dataset) para interactuar con él de todas las formas mencionadas anteriormente.

Para facilitar al usuario completar esta tarea, se desarrolló un selector de archivos que solo muestra archivos csv. En caso de que el usuario intente subir un fichero que no pueda ser soportado por la aplicación, se le notificará al instante cuál ha sido el error. 

Aun así, esta funcionalidad tiene una exigencia adicional y es que el fichero a subir ya haya sido analizado previamente, es decir, que el archivo ya cuente con el atributo estado.

Lo cual puede ser un problema si el usuario no cuenta con una aplicación como la desarrollada en este proyecto, capaz de analizar ficheros csv y añadir el atributo estado.

Es por eso que el siguiente paso en el desarrollo del proyecto fue crear la función analizar fichero.


\textbf{Analizar fichero}

Como he mencionado antes, esta función se desarrolla para complementar a la función Añadir fichero, permitiendo así a los usuarios que desean subir sus ficheros csv a la aplicación web, pero no disponen de una herramienta capaz de analizarlos y clasificarlos en reales o infectados.

En términos de funcionamiento, es muy similar a añadir fichero, permitiendo solo analizar ficheros csv que tengan la misma estructura que el fichero webhook\_dataset solo que en esta ocasión también permite que no tengan la última columna correspondiente al atributo estado ya que esta columna es la que va a añadir esta función.

Para realizar el análisis, importa la función de análisis correspondiente a la aplicación webhook y la aplica sobre todas las filas del fichero a analizar. Una vez termina, lo añade a la carpeta personal del usuario con el mismo nombre del fichero entregado pero añadiendo el sufijo \_analizado para aclarar al usuario que ese fichero ha sido generado de esta forma.


\subsubsection{Infectar}

Para terminar, la última función implementada a la aplicación es infectar, surge de plantearse la posibilidad de que el usuario no tenga los medios para crear una red LoRaWAN pero aun así quiera disfrutar de la funcionalidad de la aplicación encargada de recibir y almacenar los paquetes de datos.

Para ello se crea la posibilidad de que el usuario desde la interfaz web pueda crear sus propios paquetes de datos ya sea de forma manual o de forma aleatoria y luego enviarlos a la aplicación webhook, la cual tratará esos paquetes como si hubieran sido enviados desde TTSS y los analizará y almacenará en el archivo correspondiente al dataset (webhook\_dataset).

Permitiendo así al usuario probar el funcionamiento de la función de análisis de datos asociada al dataset.

A continuación se muestra la imagen de un paquete de datos generado aleatoriamente, listo para ser enviado.

\imagen{paquete_aleatorio}{Imagen visual de la interfaz web referente a la función infectar}{0.8}