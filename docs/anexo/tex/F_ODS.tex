\apendice{Anexo de sostenibilización curricular}

\section{Introducción}
Este anexo incluirá una reflexión personal del alumnado sobre los aspectos de la sostenibilidad que se abordan en el trabajo.

La reflexión se ha realizado según las directrices estipuladas en el documento  sobre la introducción de la sostenibilidad de la CRUE \cite{Sos}.


\section{Competencias de sostenibilidad
adquiridas}

\begin{itemize}
    \item \textbf{Contextualización crítica del conocimiento (SOS1)}: En el trabajo he procurado analizar no solo el problema técnico planteado, sino también su relación con el entorno social y su posible repercusión ambiental. Esta visión me ha ayudado a comprender cómo una solución aparentemente eficiente puede no ser sostenible si no se considera su contexto.
    \item \textbf{Uso sostenible de recursos (SOS2)}: A lo largo del trabajo he sido consciente de la necesidad de aplicar criterios de eficiencia y reducción del consumo computacional en la fase de pruebas o despliegue, cuando ha sido posible.
    \item \textbf{Participación comunitaria (SOS3)}: Durante el desarrollo del proyecto he intentado tener en cuenta las necesidades de posibles usuarios y comunidades relacionadas con el tema tratado en el proyecto. La sostenibilidad también implica diseñar soluciones accesibles, inclusivas y que respondan a demandas reales.
    \item \textbf{Principios éticos (SOS4)}: He reflexionado sobre el impacto ético de las tecnologías tratadas, evitando decisiones que pudieran contribuir a la exclusión o la injusticia social. La sostenibilidad y equidad, ha sido un criterio clave en la toma de decisiones.
\end{itemize}



\section{Conclusiones}
EL desarrollo de este proyecto no solo ha aumentado mis conocimientos técnicos sobre diversos campos de la informática, sino que además he aprendido la importancia de tener en cuenta los principios de sostenibilidad aplicados a la informática a la hora de desarrollar un proyecto.  

