\capitulo{7}{Conclusiones y Líneas de trabajo futuras}


\section{Conclusiones}
tras finalizar este proyecto y observar todo lo realizado en el mismo, puedo concluir que se han cumplido los dos objetivos generales que se tenían al comienzo del proyecto.

Se ha logrado configurar y crear una red LoRaWAN capaz de mantener un flujo real de datos y de conectar dicha red con una aplicación web capaz de recibir, analizar y almacenar esos datos para su posterior uso.

Ademas se ha logrado desarrollar una aplicación web completa, intuitiva y funcional que ayude a los usuarios no técnicos a interactuar con datos recopilados de dispositivos IoT e introducirles de forma sutil el como es el proceso de análisis y clasificación de datos con el objetivo de detectar ataques o brechas de seguridad en la red. 

Aun así creo que esa parte es la que mas margen de mejora tiene con respecto a lo desarrollado en el proyecto, ya que los métodos referentes a mejorar la seguridad de los entornos IoT es un campo de la informática que actualmente esta en auge y queda mucho por mejorar. Desde la generación de conjuntos de datos de calidad y variados para ser usados en la búsqueda de patrones o técnicas de aprendizaje automático hasta mejorar la robustez de los medios de comunicación utilizados para transmitir los paquetes de datos.

Una vez terminado el proyecto puedo afirmar que este me ha ayudado a darme cuenta de varias cosas:

La importancia de la base teórica y practica adquirida durante todo el transcurso del grado de ingeniería informática, ya que muchas cosas realizadas durante el proyecto son temas que ya había tratado con anterioridad en el grado.

La ventaja que ofrece el tener claro desde el inicio del proyecto los objetivos que deseas cumplir y las tecnologias que tienes pensdado utilizar para cumplirlos.

Por ultimo agradecer al desarrollo de este proyecto el haber aumentado mis habilidades tanto en el ámbito técnico a la hora de desarrollar software y trabajar con hardware como son los dispositivos LoRaWAN como en el ámbito organizativo a la hora de gestionar el desarrollo de un proyecto de esta magnitud y documentar todo lo relacionado con su desarrollo.


\section{Líneas de trabajo futuras}

Como ya había mencionado anteriormente y aun reconociendo los logros alcanzados con este proyecto, puedo asegurar que aun hay muchos aspectos del mismo que se pueden mejorar o nuevas funcionalidades que no se habían planteado que podría ser interesante tenerlas en cuenta:

\begin{itemize}
    \item \textbf{Mejora del dataset}:
    Ya he mencionado anteriormente los principales problemas referentes al dataset utilizado en el proyecto, por eso mismo un trabajo a futuro podría ser el mejorar la calidad del dataset generado haciéndolo mas amplio y completo. Facilitando así la tarea referente a analizar correctamente los datos recibidos.
    
    \item \textbf{Exploración de diferentes métodos de clasificación}:
    En este proyecto se ha optado por utilizar un método de clasificación que no tiene por que ser el mas adecuado para este trabajo, por ello estudiar diferentes opciones y probar si son mejores que la utilizada actualmente, mejorando así los resultados de clasificación del proyecto es una linea de trabajo a futuro completamente valida.
    
    \item \textbf{Generalización en el ámbito hardware}:
    Actualmente el proyecto desarrollado esta centrado y focalizado en dar soporte al sensor MerryIoT Motion Detection MS10, por lo tanto a futuro se podría plantear la idea de generalizar el proyecto para que sea capaz de soportar el flujo de datos de cualquier tipo de sensor IoT.
    
    \item \textbf{Análisis del flujo de datos recibido}:
    Este apartado se ha mencionado en trabajos relacionados y es la posibilidad de implementar al proyecto la capacidad de analizar el trafico de red IoT con el fin de detectar posibles vectores de ataque o comportamientos anómalos, mejorando así tanto la seguridad como el analizador del proyecto.
    
    \item \textbf{Creación de una interfaz referente a la red IoT}:
    Este apartado esta relacionado con el anterior y consiste en plantearse crear una interfaz que pueda usar el usuario, como hace en la segunda aplicación web pero referente a la primer aplicación encargada de analizar y recibir los datos, añadiendo funcionalidades adicionales como puede ser mostrar en tiempo real un grafo de nodos con el flujo de datos entre ellos.
\end{itemize}