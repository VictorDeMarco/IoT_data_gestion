\apendice{Especificación de Requisitos}

\section{Introducción}

En este apartado se describen los objetivos generales del proyecto, los requisitos tanto funcionales como no funcionales y los distintos casos de uso del software.

% Caso de Uso 1 -> Consultar Experimentos.
%\begin{table}[p]
%	\centering
%	\begin{tabularx}{\linewidth}{ p{0.21\columnwidth} p{0.71\columnwidth} }
%		\toprule
%		\textbf{CU-1}    & \textbf{Ejemplo de caso de uso}\\
%		\toprule
%		\textbf{Versión}              & 1.0    \\
%		\textbf{Autor}                & Alumno \\
%		\textbf{Requisitos asociados} & RF-xx, RF-xx \\
%		\textbf{Descripción}          & La descripción del CU \\
%		\textbf{Precondición}         & Precondiciones (podría haber más de una) \\
%		\textbf{Acciones}             &
%		\begin{enumerate}
%			\def\labelenumi{\arabic{enumi}.}
%			\tightlist
%			\item Pasos del CU
%			\item Pasos del CU (añadir tantos como sean necesarios)
%		\end{enumerate}\\
%		\textbf{Postcondición}        & Postcondiciones (podría haber más de una) \\
%		\textbf{Excepciones}          & Excepciones \\
%		\textbf{Importancia}          & Alta o Media o Baja... \\
%		\bottomrule
%	\end{tabularx}
%	\caption{CU-1 Nombre del caso de uso.}
%\end{table}
%
\section{Objetivos generales}


\begin{itemize}
    \item Diseñar e implementar una solución completa que permite recoger, analizar, almacenar y visualizar los datos generados por un dispositivo IoT conectado a una red LoRaWAN.
    \item Proporcionar a usuarios no técnicos una interfaz web intuitiva desde la que puedan interactuar con los datos, analizar su contenido y detectar posibles anomalías.
\end{itemize}
    


\section{Catálogo de requisitos}
Una vez claros los objetivos generales del proyecto podemos definir los requisitos del mismo:
\subsubsection{Requisitos funcionales}

\begin{itemize}
    \item \textbf{RF-1 Recepción de datos}: El sistema debe ser capaz de recoger, analizar y almacenar los datos recibidos:
    \begin{itemize}
        \item \textbf{RF-1.1 Recoger datos}: Recibir automáticamente los paquetes enviados por el dispositivo IoT a través de la red LoRaWAN y el webhook configurado.
        \item \textbf{RF-1.2 Analizar datos}: Evaluar los datos recibidos mediante reglas heurísticas y reglas de asociación para determinar si son reales.
        \item \textbf{RF-1.3 Almacenar datos}: Guardar los datos analizados en un archivo CSV que actúa como dataset principal.
    \end{itemize}

    \item \textbf{RF-2 Visualización de gráficas}: El usuario debe poder ver las gráficas correspondientes a los datos almacenados:
    \begin{itemize}
        \item \textbf{RF-2.1 Elegir modo de visualización}: Permitir al usuario seleccionar si que tipo de datos desea visualizar en las gráficas, si solo los datos reales, solo los datos infectados o todos los datos juntos.
        \item \textbf{RF-2.2 Interacción con las gráficas}: Mostrar gráficas interactivas que permiten al usuario ocultar el contenido de la gráfica o ver el valor correspondiente a un punto de la gráfica.
    \end{itemize}

    \item \textbf{RF-3 Gestión de archivos}: El usuario debe poder interactuar de diferentes formas con los archivos almacenados:
    \begin{itemize}
        \item \textbf{RF-3.1 Ver contenido}: Permitir la visualización del contenido de cada archivo CSV.
        \item \textbf{RF-3.2 Aplicar archivo}: Activar un archivo específico para que se visualice su contenido en las gráficas.
        \item \textbf{RF-3.3 Eliminar archivo}: Borrar un archivo CSV del sistema de almacenamiento del usuario.
        \item \textbf{RF-3.4 Descargar archivo}: Permitir al usuario descargar cualquier archivo CSV que haya subido o generado.
    \end{itemize}

    \item \textbf{RF-4 Cargar archivos}: El usuario debe ser capaz de añadir nuevos ficheros a la aplicación:
    \begin{itemize}
        \item \textbf{RF-4.1 Cargar fichero}: Subir archivos CSV que cumplan con la estructura esperada por la aplicación.
        \item \textbf{RF-4.2 Analizar fichero}: Procesar automáticamente los datos del archivo cargado para analizar su contenido y otorgar a cada fila de datos el atributo estado.
    \end{itemize}

    \item \textbf{RF-5 Simulación de envío de datos}: El usuario debe tener la opción de enviar datos al dataset:
    \begin{itemize}
        \item \textbf{RF-5.1 Envío de datos manual}: Rellenar un formulario web para generar un paquete con valores personalizados.
        \item \textbf{RF-5.2 Envío de datos aleatorio}: Generar y enviar automáticamente un paquete simulado con valores aleatorios.
    \end{itemize}

    \item \textbf{RF-6 Gestión de usuarios}: El usuario debe ser capaz de tener su propia cuenta para gestionar sus datos de forma aislada:
    \begin{itemize}
        \item \textbf{RF-6.1 Registrar usuario}: El usuario debe  poder registrar una nueva cuenta proporcionando un usuario y contraseña permitidos.
        \item \textbf{RF-6.2 Inicio de sesión}: El usuario debe poder iniciar sesión en el sistema con sus credenciales.
        \item \textbf{RF-6.3 Recuperar contraseña}: El usuario debe poder restablecer su contraseña en caso de olvido.
        \item \textbf{RF-6.4 Cierre de sesión}: El usuario debe poder cerrar su sesión de forma segura.
    \end{itemize}
 
\end{itemize}




\subsubsection{Requisitos no funcionales}

\begin{itemize}
    \item \textbf{RNF-1 Usabilidad}: La interfaz web debe ser intuitiva y fácil de usar por usuarios sin conocimientos técnicos.

    \item \textbf{RNF-2 Rendimiento}: El tiempo de respuesta del sistema para las acciones comunes (como visualizar gráficas o subir archivos)debe ser razonable.

    \item \textbf{RNF-3 Compatibilidad}: El sistema debe ser compatible con los principales navegadores modernos.

    \item \textbf{RNF-4 Seguridad}: Cada usuario debe tener acceso únicamente a sus propios archivos. La gestión de cuentas debe estar protegida mediante autenticación básica (usuario y contraseña).

    \item \textbf{RNF-5 Portabilidad}: El sistema debe estar preparado para ejecutarse dentro de contenedores Docker para facilitar su despliegue en diferentes entornos.

    \item \textbf{RNF-6 Mantenibilidad}: El código fuente debe estar organizado y documentado para permitir su mantenimiento y evolución. Además, debe estar controlado mediante un sistema de versiones.

    \item \textbf{RNF-7 Escalabilidad básica}: Aunque el sistema está diseñado para un entorno local o pequeño, debe poder ampliarse en el futuro para aceptar nuevos sensores o reglas de análisis.

\end{itemize}

\section{Especificación de requisitos}
En este apartado se especifican los distintos casos de uso de nuestro proyecto:

\begin{table}[p]
	\centering
	\begin{tabularx}{\linewidth}{ p{0.21\columnwidth} p{0.71\columnwidth} }
		\toprule
		\textbf{CU-1}    & Recepción y almacenamiento de datos \\
		\toprule
		\textbf{Versión}              & 1.0 \\
		\textbf{Autor}                & Víctor De Marco Velasco \\
		\textbf{Requisitos asociados} & RF-1.1, RF-1.2, RF-1.3 \\
		\textbf{Descripción}          & El sistema recibe automáticamente paquetes enviados por el sensor IoT, los analiza mediante reglas heurísticas y los almacena en un archivo CSV. \\
		\textbf{Precondición}         & El sensor debe estar configurado y transmitiendo datos a través de LoRaWAN y TTN. El webhook debe estar activo. \\
		\textbf{Acciones}             &
		\begin{enumerate}
			\item El sensor IoT envía un paquete a TTN.
			\item TTN reenvía el paquete al webhook configurado.
			\item El servidor Flask recibe el paquete.
			\item El sistema analiza los datos con reglas heurísticas y de asociación.
			\item El paquete se clasifica como real o infectado.
			\item El paquete se guarda en un archivo CSV local.
		\end{enumerate}\\
		\textbf{Postcondición}        & El archivo CSV es actualizado con una nueva fila correspondiente al paquete recibido y analizado. \\
		\textbf{Excepciones}          & Error en la conexión TTN o en el webhook. Paquete malformado. Problemas de almacenamiento en disco. \\
		\textbf{Importancia}          & Alta \\
		\bottomrule
	\end{tabularx}
	\caption{CU-1 Recepción y almacenamiento de datos.}
\end{table}

\begin{table}[p]
	\centering
	\begin{tabularx}{\linewidth}{ p{0.21\columnwidth} p{0.71\columnwidth} }
		\toprule
		\textbf{CU-2}    & Visualización de gráficas \\
		\toprule
		\textbf{Versión}              & 1.0 \\
		\textbf{Autor}                & Víctor De Marco Velasco \\
		\textbf{Requisitos asociados} & RF-2.1, RF-2.2, RF-6.2 \\
		\textbf{Descripción}          & El usuario visualiza datos de temperatura y humedad en gráficas interactivas a partir de los archivos CSV almacenados. \\
		\textbf{Precondición}         & El usuario debe estar autentificado. \\
		\textbf{Acciones}             &
		\begin{enumerate}
			\item El usuario accede a la sección de gráficas.
			\item Selecciona el tipo de datos (reales, infectados o todos).
			\item El sistema genera las gráficas correspondientes.
			\item El usuario puede interactuar con las gráficas: ocultarlas , ver valores exactos, etc.
		\end{enumerate}\\
		\textbf{Postcondición}        & Las gráficas se muestran en pantalla correctamente. \\
		\textbf{Excepciones}          & Archivo inválido o sin datos. Error en la carga del gráfico. \\
		\textbf{Importancia}          & Alta \\
		\bottomrule
	\end{tabularx}
	\caption{CU-2 Visualización de gráficas.}
\end{table}

\begin{table}[p]
	\centering
	\begin{tabularx}{\linewidth}{ p{0.21\columnwidth} p{0.71\columnwidth} }
		\toprule
		\textbf{CU-3}    & Gestión de archivos CSV \\
		\toprule
		\textbf{Versión}              & 1.0 \\
		\textbf{Autor}                & Víctor De Marco Velasco \\
		\textbf{Requisitos asociados} & RF-3.1, RF-3.2, RF-3.3, RF-3.4, RF-6.2 \\
		\textbf{Descripción}          & El usuario puede ver, activar, eliminar o descargar los archivos CSV que están asociados a su cuenta. \\
		\textbf{Precondición}         & El usuario debe haber iniciado sesión correctamente y debe tener archivos disponibles. \\
		\textbf{Acciones}             &
		\begin{enumerate}
			\item El usuario accede al panel de archivos.
			\item Visualiza la lista de archivos disponibles.
			\item Selecciona un archivo para:
			\begin{itemize}
				\item Ver su contenido en tabla.
				\item Aplicarlo para las gráficas.
				\item Eliminarlo definitivamente.
				\item Descargarlo localmente.
			\end{itemize}
		\end{enumerate}\\
		\textbf{Postcondición}        & El sistema actualiza el estado del archivo según la acción seleccionada. \\
		\textbf{Excepciones}          & Archivo inexistente, error al acceder al almacenamiento o permisos incorrectos. \\
		\textbf{Importancia}          & Alta \\
		\bottomrule
	\end{tabularx}
	\caption{CU-3 Gestión de archivos CSV.}
\end{table}

\begin{table}[p]
	\centering
	\begin{tabularx}{\linewidth}{ p{0.21\columnwidth} p{0.71\columnwidth} }
		\toprule
		\textbf{CU-4}    & Carga de archivos CSV \\
		\toprule
		\textbf{Versión}              & 1.0 \\
		\textbf{Autor}                & Víctor De Marco Velasco \\
		\textbf{Requisitos asociados} & RF-4.1,RF-6.2 \\
		\textbf{Descripción}          & El usuario sube un archivo CSV y el sistema lo almacena en su lista personal". \\
		\textbf{Precondición}         & El usuario debe haber iniciado sesión. El archivo debe cumplir con el formato requerido. \\
		\textbf{Acciones}             &
		\begin{enumerate}
			\item El usuario accede a la sección de carga.
			\item Selecciona o arrastra un archivo .csv.
			\item El sistema valida el archivo.
			\item El archivo queda almacenado en su cuenta.
		\end{enumerate}\\
		\textbf{Postcondición}        & El archivo queda almacenado y disponible para ser utilizado por el usuario. \\
		\textbf{Excepciones}          & Formato incorrecto o archivo corrupto. \\
		\textbf{Importancia}          & Alta \\
		\bottomrule
	\end{tabularx}
	\caption{CU-4 Carga de archivos CSV.}
\end{table}

\begin{table}[p]
	\centering
	\begin{tabularx}{\linewidth}{ p{0.21\columnwidth} p{0.71\columnwidth} }
		\toprule
		\textbf{CU-5}    & Análisis de archivos CSV \\
		\toprule
		\textbf{Versión}              & 1.0 \\
		\textbf{Autor}                & Víctor De Marco Velasco \\
		\textbf{Requisitos asociados} & RF-4.2,RF-6.2 \\
		\textbf{Descripción}          & El usuario sube un archivo CSV, el sistema lo analiza añadiendo la columna estado y luego lo almacena en su lista personal". \\
		\textbf{Precondición}         & El usuario debe haber iniciado sesión. El archivo debe cumplir con el formato requerido. \\
		\textbf{Acciones}             &
		\begin{enumerate}
			\item El usuario accede a la sección de carga.
			\item Selecciona o arrastra un archivo .csv.
			\item El sistema valida el archivo.
            \item El sistema analiza los datos y genera la columna “estado”.
			\item El archivo queda almacenado en su cuenta.
		\end{enumerate}\\
		\textbf{Postcondición}        & El archivo queda analizado y disponible para ser utilizado por el usuario. \\
		\textbf{Excepciones}          & Formato incorrecto o archivo corrupto. \\
		\textbf{Importancia}          & Alta \\
		\bottomrule
	\end{tabularx}
	\caption{CU-5 Análisis de archivos CSV.}
\end{table}

\begin{table}[p]
	\centering
	\begin{tabularx}{\linewidth}{ p{0.21\columnwidth} p{0.71\columnwidth} }
		\toprule
		\textbf{CU-6}    & Simulación de envío de datos \\
		\toprule
		\textbf{Versión}              & 1.0 \\
		\textbf{Autor}                & Víctor De Marco Velasco \\
		\textbf{Requisitos asociados} & RF-5.1, RF-5.2,RF-6.2 \\
		\textbf{Descripción}          & El usuario puede simular el envío de paquetes, bien manualmente introduciendo los datos, o automáticamente con valores aleatorios. \\
		\textbf{Precondición}         & El usuario debe estar autenticado. \\
		\textbf{Acciones}             &
		\begin{enumerate}
			\item El usuario accede a la sección de simulación.
			\item Selecciona si quiere enviar datos manuales o aleatorios.
			\item Si elige manual, introduce valores personalizados.
			\item Si elige aleatorio, el sistema genera los datos automáticamente.
			\item El paquete se analiza y se guarda en el dataset correspondiente.
		\end{enumerate}\\
		\textbf{Postcondición}        & El archivo CSV se actualiza con el nuevo paquete generado. \\
		\textbf{Excepciones}          & Valores inválidos, error al generar datos, fallo de almacenamiento. \\
		\textbf{Importancia}          & Media \\
		\bottomrule
	\end{tabularx}
	\caption{CU-6 Simulación de envío de datos.}
\end{table}

\begin{table}[p]
	\centering
	\begin{tabularx}{\linewidth}{ p{0.21\columnwidth} p{0.71\columnwidth} }
		\toprule
		\textbf{CU-7}    & Acceso a la aplicación web \\
		\toprule
		\textbf{Versión}              & 1.0 \\
		\textbf{Autor}                & Víctor De Marco Velasco \\
		\textbf{Requisitos asociados} & RF-6.1, RF-6.2,RF-6.3 \\
		\textbf{Descripción}          & El usuario debe poder acceder a la aplicación web con una cuenta personal si así lo desea . \\
		\textbf{Precondición}         & \\
		\textbf{Acciones}             &
		\begin{enumerate}
			\item El usuario accede a la dirección de la aplicación web.
			\item El usuario registra una cuenta si no dispone de una.
			\item El usuario intenta recuperar la contraseña si no recuerda cual uso para registrarse.
			\item El usuario procede a iniciar sesión con el usuario y contraseña que uso para registrarse.
		\end{enumerate}\\
		\textbf{Postcondición}        & El sistema guarda sus credenciales y le permite acceder a la aplicación. \\
		\textbf{Excepciones}          & Credenciales incorrectas o campos vacíos. \\
		\textbf{Importancia}          & Alta \\
		\bottomrule
	\end{tabularx}
	\caption{CU-7 Acceso a la aplicación web.}
\end{table}