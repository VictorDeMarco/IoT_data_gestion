\apendice{Plan de Proyecto Software}

\section{Introducción}
En este apartado se explica la planificación temporal seguida durante el proyecto y el estudio de cuán viable es tanto en el ámbito económico como legal.
	
\section{Planificación temporal}
La planificación temporal del proyecto se ha llevado a cabo utilizando Sprints haciendo referencia a la metodología Scrum. La duración, el objetivo y cómo cumplir dicho objetivo en cada Sprint deben decidirse entre los integrantes que van a formar parte del proyecto antes de comenzar el Sprint.
Cuando llega la fecha límite acordada se comprueba si ha sido posible realizar todo lo planificado durante la reunión de creación del Sprint. Según lo que se haya logrado realizar se actualizan los objetivos y tareas del proyecto. Esto continúa hasta que se da por finalizado el proyecto.
\section{Estudio de viabilidad}
El estudio de viabilidad del proyecto se ha realizado desde el punto de vista económico y legal.
\subsection{Viabilidad económica}
Para poder determinar si un proyecto software es viable económicamente, se deben calcular los gastos que supondría realizar el proyecto y los beneficios que se podrían obtener del mismo. Una vez realizados estos cálculos se puede determinar si el proyecto es un proyecto viable desde el punto de vista económico.
\subsubsection{Costes Hardware}
Para realizar este proyecto se ha necesitado:
\begin{itemize}
    \item Ordenador portátil Asus: 900€
    \item Dragino LoRaWAN Gateway: 165€
    \item Merry IoT Motion Detection: 45€
\end{itemize}
Teniendo en cuenta la vida útil económica de estos dispositivos, el coste anual de cada uno sería:
\begin{itemize}
    \item Ordenador portátil Asus (4 años): 225€
    \item Dragino LoRaWAN Gateway (5 años): 33€
    \item Merry IoT Motion Detection (6 años): 7,5€
\end{itemize}
Aunque estos dispositivos seguramente podrían seguir usándose pasada la vida útil contemplada, debido a posibles mejoras técnicas y  nuevas tecnologías superiores a estos dispositivos convendría actualizarlos y prescindir de ellos. 
\subsubsection{Costes Software}
Para realizar este proyecto se ha necesitado:
\begin{itemize}
    \item Sistema operativo Windows 11: 145€
    \item Dominio web: 4,55€
\end{itemize}
Aclarar que la compra del dominio web se paga de forma anual y el sistema operativo se amortiza dependiendo del dispositivo en el que este instalado, por lo tanto dispone de 4 años de vida útil al estar en el ordenador portátil Asus.
\subsubsection{Coste de Desarrollo}
Considerando que el proyecto lo he desarrollado yo solo el único coste de desarrollo a tener en cuenta seria mi sueldo.
\begin{itemize}
    \item Sueldo anual: 6.240€
    \item Sueldo mensual: 520€
\end{itemize}
Este sueldo esta calculado suponiendo que se trabaja 40 horas mensuales.
\subsubsection{Costes Adicionales}
De momento no se ha necesitado añadir ningún coste adicional no planteado anteriormente.
\subsubsection{Coste Total}
Suponiendo que se llevara a cabo el proyecto durante un año vamos a calcular el coste del mismo:
\begin{table}[H]
\centering
\begin{tabular}{|l|c|c|p{5cm}|}
\hline
\textbf{Elemento} & \textbf{Coste Anual (€)} & \textbf{Comentario} \\
\hline
Hardware: Portatil Asus & 225  & Vida útil 4 años. \\
\hline
Hardware: Gateway Dragino & 33  & Vida útil 5 años. \\
\hline
Hardware: Sensor MerryIoT & 7,5  & Vida útil 6 años. \\
\hline
Software: Dominio web & 4,55 & Uso exclusivo para el proyecto. \\
\hline
Software: Windows 11 & 36,25  & Licencia del sistema operativo \\
\hline
Horas de desarrollo & 6240  & Sueldo aproximado \\
\hline
\textbf{TOTAL} & 6.546,3  & Coste total de un año de desarrollo \\
\hline
\end{tabular}
\caption{Resumen de costes estimados durante un año}
\label{tab:viabilidad-economica}
\end{table}
Una vez observados los cálculos pretender sacar rédito económico del proyecto parece no ser muy realista ya que es un proyecto que individualmente no tiene mucho interés comercial, seria mas como parte de un proyecto mucho mas grande. 

\subsection{Viabilidad legal}
Para poder determinar si un proyecto es viable legalmente se deben revisar las licencias utilizadas durante el proyecto y determinar si hay alguna barrera legal que impida el desarrollo y uso del proyecto.


\subsubsection{Protección de Datos}

El proyecto desarrollado recoge datos procedentes de un sensor IoT que, si bien no identifican directamente a personas, podrían considerarse datos personales en determinadas circunstancias si permiten inferir comportamientos (por ejemplo, detección de presencia o hábitos). Por eso mismo en caso de ampliarse el uso de esta solución en entornos reales con usuarios, sería necesario cumplir con los reglamentos y leyes referentes a la protección de datos del lugar donde se fuera a usar:


\begin{itemize}
    \item Garantizar que la procedencia de los datos sea anónima.
     \item Limitar el almacenamiento y uso a los fines autorizados.
     \item Disponer de medidas de seguridad técnicas y organizativas para proteger los datos.
\end{itemize}


Además, si el usuario utiliza las funciones referentes a la recepción de paquetes enviados desde un servidor web, se expone a estar conectado a la red pudiendo comprometer su privacidad.



\subsubsection{Licencias y uso de software}

El proyecto utiliza las siguientes tecnologías software:

\begin{table}[H]
\centering
\begin{tabular}{|l|c|l|}
\hline
\textbf{Herramienta} & \textbf{Versión} & \textbf{Licencia} \\
\hline
Python & 3.11.0 & PSF \\
Flask & 2.3 & BSD-3-Clause \\
Pandas & 2.3.0 & BSD-3-Clause \\
SQLAlchemy & 2.0.41 & MIT \\
Werkzeug & 2.3 & BSD-3-Clause \\
Requests & 2.32.4 & Apache 2.0 \\
Pytz & 2025.2 & MIT \\
mlxtend & 0.23.4 & BSD-3-Clause \\
IntelliJ IDEA & 2024.3,6 & Apache 2.0 \\
Docker & 28.0.4 & Apache 2.0 \\
Docker Desktop & 4.40.0 & Propietaria \\
Git & 2.45.1 & GPLv2 \\
GitHub & N/A & Propietaria \\
Overleaf & N/A & Propietaria \\
Draw.io & 27.2.0 & Apache 2.0 \\
HTML & 5 & BSD 3-Clause \\
Bootstrap 5 & 5.3.0 & MIT \\
Chart.js & 4.5.0 & MIT \\
\hline
\end{tabular}
\caption{Licencias de herramientas utilizadas en el proyecto}
\end{table}


Todos los recursos software utilizados permiten su uso académico sin restricción y se respetan las licencias correspondientes.

\subsubsection{Uso de hardware}

El hardware utilizado incluye un gateway LoRaWAN Dragino y un sensor MerryIoT de detección de movimiento y condiciones ambientales. Ambos dispositivos se han instalado en mi domicilio por lo tanto no se ha realizado ninguna instalación en un espacio público ni se requiere certificación adicional, ya que ambos operan en bandas ISM de uso libre y conforme a normativa europea.

\subsubsection{Conclusión}

En conclusión  el proyecto es legalmente viable, no vulnera la normativa de protección de datos ni derechos de terceros, y respeta las condiciones de uso de todo el software y hardware utilizado.



