\capitulo{4}{Técnicas y herramientas}
Esta parte de la memoria tiene como objetivo presentar las técnicas metodológicas y las herramientas de desarrollo que se han utilizado para llevar a cabo el proyecto.

\section{Técnicas utilizadas}
\subsection{SCRUM}
SCRUM es un marco de trabajo ágil utilizado principalmente en el desarrollo de software, divide el trabajo en ciclos cortos y regulares llamados "sprints", normalmente de 1 a 4 semanas, en los que se desarrolla un incremento funcional del producto.

Para obtener mas información sobre la metodología Scrum utilizada durante el proyecto consultar el apartado A.2.

\section{Herramientas utilizadas}
\subsection{Hardware}
\subsubsection{Gateway Dragino LPS8N}
\cite{Haw:Dra}
\subsubsection{Sensor MerryIoT MS10 Motion Detection}

\subsection{Servicios en la nube y conectividad}
\subsubsection{The Things Network (TTN)/The Things Stack}
TTN es una infraestructura abierta para redes LoRaWAN. Permite registrar dispositivos, configurar gateways y gestionar los datos recibidos mediante webhooks, que envían automáticamente los paquetes recibidos hacia una URL determinada.
The Things Stack~\cite{TTN:TTS} es un servidor LoRaWAN de nivel empresarial (que incluye tanto las funciones de Servidor de red como de Servidor de aplicaciones mencionadas en la arquitectura de referencia de LoRaWAN). Además, The Things Stack incluye servicios y herramientas para gestionar de forma segura millones de dispositivos LoRaWAN en entornos de producción.
\subsubsection{Cloudflare Tunnel}

\subsection{Entorno de desarrollo}

\subsubsection{IntelliJ IDEA}
\subsubsection{Git + GitHub}

\subsection{Backend}

\subsubsection{Python 3.10}
\subsubsection{Flask}
\subsubsection{SQLAlchemy}
\subsubsection{Pandas}
\subsubsection{Werkzeug}

\subsection{Frontend}

\subsubsection{HTML5 y Jinja2}
\subsubsection{Bootstrap 5}
\subsubsection{Chart.js}
\subsubsection{JavaScript}

\subsection{Formato de datos}

\subsubsection{CSV}